%    Documentation for PRU ADC Project
%    Copyright (C) 2016  Gregory Raven
%
%    This program is free software: you can redistribute it and/or modify
%    it under the terms of the GNU General Public License as published by
%    the Free Software Foundation, either version 3 of the License, or
%    (at your option) any later version.
%
%    This program is distributed in the hope that it will be useful,
%    but WITHOUT ANY WARRANTY; without even the implied warranty of
%    MERCHANTABILITY or FITNESS FOR A PARTICULAR PURPOSE.  See the
%    GNU General Public License for more details.
%
%    You should have received a copy of the GNU General Public License
%    along with this program.  If not, see <http://www.gnu.org/licenses/>.

\chapter{Shell Scripts}

\section{RemoteProc ``Bind'' and ``Unbind'' Commands}

There are two very important shell scripts located in the shell\_scripts of the git repository.

These scripts are very simple and each contain only a single command.

The commands are described in the notes file from this github repository:

\url{https://github.com/ZeekHuge/BeagleScope}

And specifically, this is the path to the notes file:

\url{https://github.com/ZeekHuge/BeagleScope/blob/port_to_4.4.12-ti-r31%2B/docs/current_remoteproc_drivers.notes} 
	
	The commands are seen in section 2:
	
	\begin{verbatim}
	echo "4a334000.pru0" > /sys/bus/platform/drivers/pru-rproc/unbind
	echo "4a334000.pru0" > /sys/bus/platform/drivers/pru-rproc/bind
	echo "4a338000.pru1"  > /sys/bus/platform/drivers/pru-rproc/unbind
	echo "4a338000.pru1" > /sys/bus/platform/drivers/pru-rproc/bin
	\end{verbatim}
	
	The above shell commands show how the PRUs can ``bind'' and ``unbind'' from the remoteproc driver.  These commands are extremely useful and their placement in shell scripts allows them to be easily run at the command line by entering ``prumodin'' or ''prumodout''.
	
	The shell scripts should be copied to /usr/bin to make them available from any shell.
	
\section{Environment Variables and Universal IO Configuration Script}

The make process which compiles the PRU firmwares requires an environment variable to be set:

\begin{verbatim}
export PRU_CGT=/usr/share/ti/cgt-pru
\end{verbatim}

This is best done by writing this export statement plus other configurations to a file.
This file is sourced from the .bashrc file upon entering the home directory bash shell.

Other methods of setting the environment variables were tried. Using the .bashrc file was found to be the most practical.

When using the ``sudo'' command to gain temporary root access, it is necessary to use the -E option in order to preserve the environment variables.  For example:

\begin{verbatim}
sudo -E make
\end{verbatim}

will preserve the \$PRU\_CGT environment variable and will allow the successful compilation of the PRU firmwares.

The file ``pru\_gpio\_config'' is included in the shell\_scripts directory.  Source the file by adding this line to the end of the .bashrc file located in the user home directory:

\begin{verbatim}
source /home/debian/pru-pid-motor/software/shell_scripts/pru_gpio_config
\end{verbatim}

The path shown above assumes the project was cloned to the user's home directory (debian by default in the latest distribution).
