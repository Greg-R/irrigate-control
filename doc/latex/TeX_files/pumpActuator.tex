%    Documentation for Irrigation Control Project
%    Copyright (C) 2017  Gregory Raven
%
%    This program is free software: you can redistribute it and/or modify
%    it under the terms of the GNU General Public License as published by
%    the Free Software Foundation, either version 3 of the License, or
%    (at your option) any later version.
%
%    This program is distributed in the hope that it will be useful,
%    but WITHOUT ANY WARRANTY; without even the implied warranty of
%    MERCHANTABILITY or FITNESS FOR A PARTICULAR PURPOSE.  See the
%    GNU General Public License for more details.
%
%    You should have received a copy of the GNU General Public License
%    along with this program.  If not, see <http://www.gnu.org/licenses/>.

\chapter{Pump Actuator Class}

The pumpActuator class models the control functions of the irrigation system.  
It extends the Node.js EventEmitter and thus it can emit events.

The class maintains a data structure which stores the current state of the 
system.  This data structure is set up by the class constructor:

\begin{verbatim}
        this.pumpMap = new Map([['pumpmotor', 0],
                              ['zone1', 0],
                              ['zone2', 0]]);
\end{verbatim}

This is an ES6 ``Map'' which is an associative array.  There is a ``key'' for 
the each pump element, and the value of 0 or 1 represents the on or off state.

As explained in the chapter on ES6 a Proxy object is used to implement the 
``Observer'' pattern.

The Proxy can observe changes made to the data structure and execute 
side-effects as required.  In this case, the GPIO state needs to be changed 
each time the pumpMap data structure is updated.  The update can happen via the 
manual controls on the web page, or they can be initiated by the Scheduler 
Object.

The system changes the state of the GPIOs and also updates the status 
indicators on the web page.  This action happens merely by writing to the 
pumpMap data structure.  So it does not matter who does the update; the 
side-effects will always be taken care of automatically.

The class contains another Map which are the paths to the virtual file system 
sys controls for the GPIOs.

The class emits a ``pumpStatusMessage'' whenever a write is done to the pumpMap.