%    Documentation for PRU ADC Project
%    Copyright (C) 2016  Gregory Raven
%
%    This program is free software: you can redistribute it and/or modify
%    it under the terms of the GNU General Public License as published by
%    the Free Software Foundation, either version 3 of the License, or
%    (at your option) any later version.
%
%    This program is distributed in the hope that it will be useful,
%    but WITHOUT ANY WARRANTY; without even the implied warranty of
%    MERCHANTABILITY or FITNESS FOR A PARTICULAR PURPOSE.  See the
%    GNU General Public License for more details.
%
%    You should have received a copy of the GNU General Public License
%    along with this program.  If not, see <http://www.gnu.org/licenses/>.

\chapter{Ecmascript 6}

The following describes the simplest possible set-up.  Everything was done via the command line, and the vim editor was used extensively to develop the C code and shell scripts.

SSH was used to remotely access the BBG from a 64 bit desktop computer running Ubuntu 16.04.

For reference, here is the link to the TI PRU support package:

\url{https://git.ti.com/pru-software-support-package}

The above package can be cloned to the BBG.  There is a good set of examples and labs included.  The labs are documented here:

\url{http://processors.wiki.ti.com/index.php/PRU_Training:_Hands-on_Labs}

Note that the files appropriate for the BBG are in the folders with name am335x.

The Makefiles in the labs and examples were designed to work with a particular set-up which can be easily implemented on the BBG.

The following is a list of recommended steps to prepare a BBG for compiling PRU C files.
This process assumes a recent SD card image (IOT recommended) which is loaded with the PRU compiler (clpru) and libraries.  Another assumption is that the RemoteProc and RPMsg kernel drivers are included and that they are loaded during the start-up process.  This is true for some, but not all, recently published images as of December, 2016.

\section{Activate RemoteProc PRU and Kernel Modules}

The newest Beaglebone Debian distributions do not have the Remoteproc framework activated by default!

The following process activates the framework which includes several loadable kernel modules.  This is a prerequisite for the remainder of the setup process.

This process was tested using this image:

bone-debian-8.6-iot-armhf-2016-10-30-4gb.img

The ``IOT'' (Internet Of Things) image includes the set of tools required to install and compile the required software.
The image was found at this web site:

\url{http://elinux.org/Beagleboard:BeagleBoneBlack_Debian#microSD.2FStandalone:_.28iot.29_.28BeagleBone.2FBeagleBone_Black.2FBeagleBone_Green.29}

The IOT distribution includes very useful scripts in the following directory:

/opt/scripts/tools

The script grow\_partition.sh will expand the file system on the micro-sd card to its full capacity.  Running this script is highly recommended before proceeding with this process!

\section{Activate Remoteproc: Step-by-step Process}

\begin{enumerate}
\item  Write Beaglebone image to micro-sd and expand partition as required.
\item  Insert micro-sd into BBG slot, press boot and power buttons and release.  Non-flasher images may not require the boot button to be pressed, and the board will boot and run from the micro-sd card automatically.
\item  ssh debian@192.168.1.7 (or whatever the IP is set to).  If you are using a router with a GUI control application, it may have a display which indicates the board is connected and to which IP address has been assigned to it.
\item  Execute
\begin{verbatim}
sudo apt-get update
\end{verbatim}
\item  Execute
\begin{verbatim}
uname -r
\end{verbatim} 
to verify kernel version.  Please note that the RemoteProc framework is still evolving and it is recommended to verify that the kernel used will work with the PRU support package examples.

The rest of the set-up will be completed using root access.
Execute
\begin{verbatim}
sudo su
\end{verbatim}
and authenticate if required to switch to root user.
\item Clone this repository to a convenient directory (recommended /opt/scripts/tools):

\begin{verbatim}
git clone https://github.com/RobertCNelson/dtb-rebuilder
\end{verbatim}

\item Execute:
\begin{verbatim}
cd dtb-rebuilder/ 
cd src/arm
\end{verbatim}
\item Find and edit the top of the device tree dts file.
For BBG, this is:
\begin{verbatim}
am335x-bonegreen.dts
\end{verbatim}
Uncomment this single line in the file:
\begin{verbatim}
/*   #include "am33xx-pruss-rproc.dtsi"  */
\end{verbatim}

The line should now look like this:
\begin{verbatim}
#include "am33xx-pruss-rproc.dtsi"
\end{verbatim}

A new include statement must be added to the same file for configuration of the Quadrature Decoder and the PRU PWM.  This file must be copied from the repository:

\begin{verbatim}
pru-pid-motor/software/dtsi/am335x-boneblack-prupid.dtsi
\end{verbatim}

into the arm directory in the dtb-rebuilder.

Add this line to the end of the am335x-bonegreen.dts file:

\begin{verbatim}
#include "am335x-boneblack-prupid.dtsi"
\end{verbatim}
 
Save and exit.
\item
Execute:
\begin{verbatim}
cd /etc/modprobe.d
\end{verbatim}
Create a new file named:
\begin{verbatim}
pruss-blacklist.conf
\end{verbatim}
Note that the latest images may already have this file.

Add this single line to the file:
\begin{verbatim}
blacklist uio_pruss
\end{verbatim}
Save and exit.
\item
cd back to the dtb-rebuilder directory.  Execute these commands:
\begin{verbatim}
make 
make install 
reboot
\end{verbatim} 
\end{enumerate}
To verify that the above process was successful:

\begin{verbatim}
cd /sys/bus/platform/devices
ls
\end{verbatim}

Now look for the following in the output from the ls command:
\begin{verbatim}
4a300000.pruss
4a320000.intc
4a334000.pru0
4a338000.pru1
\end{verbatim}

The appearance of the above entries indicates that the RemoteProc PRU activation process was successful.

\section{PRU Compiler Setup Process}

\begin{enumerate}
\item  Execute these commands:
\begin{verbatim}
cd /
\end{verbatim} and then 
\begin{verbatim}
find . -name cgt-pru
\end{verbatim}

The path should be something like 
\begin{verbatim}
/usr/share/ti/cgt-pru
\end{verbatim}  

This is the location of the PRU library and includes.
However, the clpru compiler binary is not located in this directory.  Run this command:
\begin{verbatim}
which clpru
\end{verbatim}
and the result will be something like:
\begin{verbatim}
/usr/bin/clpru
\end{verbatim}
This is the path to the compiler binary.

The PRU C compiler needs to find the include and lib directories.  The Makefiles in the PRU Support Package look for the compiler binary at this path, so the following changes must be made.

Execute the following commands (as root):
\begin{verbatim}
cd /usr/share/ti/cgt-pru
mkdir bin
cd bin
ln -s /usr/bin/clpru clpru
\end{verbatim}
So now the Makefiles will find the compiler executable in the correct location via the link.
\item  Now install the PRU Support Package in the home directory:

\begin{verbatim}
cd /home/debian
git clone git://git.ti.com/pru-software-support-package/pru-software-support-package.git
\end{verbatim}

This will clone a copy of the latest pru support package.
\item  cd into lab\_5 in the package and execute the make command:
\begin{verbatim}
cd pru-software-support-package/labs/lab_5/solution/PRU_Halt
make
\end{verbatim}
This will fail, as the Makefile is looking for environment variable \$PRU\_CGT.  Execute:

\begin{verbatim}
export PRU_CGT=/usr/share/ti/cgt-pru
\end{verbatim}

Now execute the make command again.  It should succeed.  A new directory ``gen'' should appear.  The above command is included in the file ``pru\_gpio\_config'' in the shell\_scripts directory of the project's Github repository.

\item  Execute the following:
\begin{verbatim}
cd gen
cp PRU_Halt.out am335x-pru0-fw
cp am335x-pru0-fw /lib/firmware
\end{verbatim}
This renames the executable binary and copies it to the directory at which Remoteproc expects to find PRU firmwares.
\item  Now cd into the PRU\_RPMsg\_Echo\_Interrupt1 directory in the same lab\_5.
Edit main.c as follows:
\begin{verbatim}
//#define CHAN_NAME					"rpmsg-client-sample"
#define CHAN_NAME					"rpmsg-pru"
\end{verbatim}
The ``CHAN\_NAME'' define is now set to ``rpmsg-pru''.
\item  Now execute almost the same as \#9, except this time the firmware for PRU1 is compiled a copied:
\begin{verbatim}
make
cd gen
cp PRU_RPMsg_Echo_Interrupt1.out am335x-pru1-fw
cp am335x-pru1-fw /lib/firmware
\end{verbatim}
The compilation of both PRU firmwares are complete and they are copied to /lib/firmware.
\item  reboot
\item  Execute:

\begin{verbatim}
lsmod
\end{verbatim}

These kernel modules should be present in the output:

\begin{verbatim}
pru_rproc              15431  0 
pruss_intc              8603  1 pru_rproc
pruss                  12026  1 pru_rproc
\end{verbatim}
Now execute:
\begin{verbatim}
rmmod pru_rproc
modprobe pru_rproc
\end{verbatim}

The rmmod command removes the remoteproc module pru\_rproc.
The modprobe command re-inserts the same module.
\item  
\begin{verbatim}
cd /dev
ls
\end{verbatim} 

Look for rpmsg\_pru31 character device file.  It will be there!
This confirms that the PRU C compiler are properly configured and that the Makefile for this project will compile the PRU binaries successfully.  The RemoteProc drivers are also tested with the successful completion of the above set-up process.  This is not the entire configuration, however.
\end{enumerate}

\section{Additional Configuration Required to Compile the PRU Remoteproc Project}

Some components of the PRU Support Package need to be added to the PRU C compiler directories.  The Makefile expects to find these components in these directories.

Execute these commands in the directory in which the PRU Support Package was cloned:

\begin{verbatim}
cd pru-software-support-package/
cp -r include $PRU_CGT/includeSupportPackage
cp lib/rpmsg_lib.lib $PRU_CGT/lib
\end{verbatim}

This should complete the configuration required to run the project's Makefile successfully.



