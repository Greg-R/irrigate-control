%    Documentation for PRU ADC Project
%    Copyright (C) 2016  Gregory Raven
%
%    This program is free software: you can redistribute it and/or modify
%    it under the terms of the GNU General Public License as published by
%    the Free Software Foundation, either version 3 of the License, or
%    (at your option) any later version.
%
%    This program is distributed in the hope that it will be useful,
%    but WITHOUT ANY WARRANTY; without even the implied warranty of
%    MERCHANTABILITY or FITNESS FOR A PARTICULAR PURPOSE.  See the
%    GNU General Public License for more details.
%
%    You should have received a copy of the GNU General Public License
%    along with this program.  If not, see <http://www.gnu.org/licenses/>.

\chapter{Ecmascript 6}

This page shows which Ecmascript 6 features are implemented in major versions
of Node:

\url{http://node.green/}

Javascript has become a feature rich and large language!
Some of the ES6 constructs used in this project:

\begin{itemize}
\item class
\item Map
\item arrow function
\item Proxy
\item const and let
\end{itemize}

A good reference for ES6 is at the Mozilla Developer Network:

\url{https://developer.mozilla.org/en-US/docs/Web/JavaScript}

A quick summary of the ES6 constructs follows.

\subsection{class}

The new ``class'' keyword does not provide new functionality.  What is does is 
allow a Javascript class object to be written in a more traditional 
object-oriented style.  This does make things easier for a person used to other 
language's syntax for defining classes.

\subsection{Map}

The ``Map'' is a new data structure object which is similar to what is called a 
``hash'', dictionary, or associative array.  Prior to ES6, the Javascript 
object was used functionally as a Map, however, this was kind of a hack.

In this project, the Map object is used in the pumpActuator object.

\subsection{Arrow Function}

``Arrow Functions'' are a simplification of the syntax used when defining a 
function.  However, they are also important in determining the the scope of 
``this'' within the function.  The arrow functions capture the ``this'' of the 
enclosing scope.  This was found to be convenient in the design of the 
pumpActuator class.

\begin{verbatim}
        this.pumpMap = new Map([['pumpmotor', 0],
                              ['zone1', 0],
                              ['zone2', 0]]);
\end{verbatim}

The ``pumpMap'' is a simple data structure to store the state of the pumpmotor 
and the two zone actuators.

\subsection{Proxy and Object.assign}

Of the above ES6 constructs, the only one requiring detailed explanation is the 
``Proxy''.  The construct is used to implement the software ``Observer'' 
pattern.

The instantiation of the ``pumpMapProxy'' is done in the constructor of the 
pumpActuator class:

\begin{verbatim}
        this.pumpMapProxy = new Proxy(this.pumpMap, this.pumpObserver());
\end{verbatim}

The Proxy's constructor takes two arguments, which in this case is this.pumpMap 
and this.pumpObserver().  The second argument requires some explanation.


\begin{verbatim}
    pumpObserver() {
        return {
            set: (target, property, value, receiver) => {
                console.log(`Setting ${property} to ${value}.`);
                this.pumpControl(property, value);
                target[property] = value;
                return true;
            }
        };
    }
\end{verbatim}

The above class method is a little bizarre.  This method merely returns a 
Javascript object.  The object in this case has a single key ``set'' and the 
value is an arrow function with four parameters: target, property, value, and 
receiver.

The Proxy creates a sort of ``watcher'' or ``observer'' of pumpMap.  The proxy 
intercepts changes written to or read from the target object (the first 
parameter of the Proxy constructor).

Using the key ``set'' causes the function (the set key's value) to be executed 
when the value is written to.  Note that the function has access to the 
intercepted property and value, and these are used in call the function 
``this.pumpControl''.  This does the physical setting of the GPIOs.  The data 
structure is also updated (target[property]=value) and ``true'' is returned to 
indicate a successful set.

The proxy does a sort of ``intercept'' of writes to the object and then can 
perform custom actions based on the the function assigned to ``set''.  Similar 
functionality for reads can be done with the ``get'' key.  The object can 
contain both and custom read and write functions can be used.  This is very 
powerful!

Functionally what the Proxy does is intercept the write to the data structure 
which stores the state of the pumpmotor and the zone solenoids.  The intercept 
runs the ``set'' function and changes the physical state of the GPIOs.

The write to the data structure is done by the server in file 
websocketserver.js:

\begin{verbatim}
           Object.assign(pumpObject.pumpHashProxy, controlObject);
\end{verbatim}

The ``controlObject'' in this case is an incoming WebSocket message (using JSON 
notation) from the browser which looks like this example:

\begin{verbatim}
{"pumpmotor":0}
\end{verbatim}

The write to the data structure in the pumpActuator object is done by the 
server in file websocketserver.js:

\begin{verbatim}
           Object.assign(pumpObject.pumpMapProxy, controlObject);
\end{verbatim}

The ``Object.assign'' is a shortcut which simply overwrites the value in 
pumpMapProxy with the value from controlObject.  The Proxy intercepts this 
write, and then executes the custom set function.

The Proxy allows a custom behavior to be executed when a data structure is 
written to or read from.  In this particular project 
with only three controls it does not standout, however, this sort of 
``Observer'' pattern is very scalable and could be very advantageous in a much 
larger and more complex system.

\subsection{const and let}

const creates a read-only reference to a block-scoped value.
let is also block-scoped, but it is a variable.

let and const solve the crazy problem of hard to understand scope and 
``hoisting'' of the var type variables of previous versions of Javascript.
 var was not used anywhere in this project.
 
 const and let are a huge improvement to Javascript!
 
 \section{Upgrading the BBGW to Node Version 8}
 
 Follow these instructions at the Node.js web site:
 
 \url{https://nodejs.org/en/download/package-manager/#debian-and-ubuntu-based-linux-distributions}



