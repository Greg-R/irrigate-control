\chapter{Setting up a systemd Irrigation Service}

systemd is responsible for booting up the user space in Debian (and other) 
GNU/Linux distributions including the BBGW.

This system can be used to start the Node.js server as an ``irrigation'' 
service.

This is easy to accomplish and can be done by creating a single text file:

\begin{verbatim}
/etc/systemd/system/irrigation.service
\end{verbatim}

The file is located in the git repository systemd folder.
Here is the contents of the file irrigation.service:

\begin{verbatim}
[Unit]
Description=Irrigation Control Server

[Service]
ExecStart=/usr/bin/node /home/debian/irrigate-control/software/node/server.js

[Install]
WantedBy=graphical.target
\end{verbatim}

The [Unit] section provides a short description of the service which is printed 
out when the service is interrogated.

The [Service] section is the complete path to the node command followed by the 
path to the server.js file.  This is the ``service'' which will be 
``daemonized'' at boot.

The [Install] section indicates the default state in which the service should 
be started.  The default state can be found by using this command:

systemctl get-default

In the case of the IOT distribution used by this project, the response is:

\begin{verbatim}
graphical.target
\end{verbatim}

Once the service unit file is in place, enable the service like this:

\begin{verbatim}
systemctl enable irrigation
\end{verbatim}

The irrigation service will now start at boot time!  Set a bookmark in your 
browser, and simply click to go straight to the irrigation control page.

To permanently disable the service:

\begin{verbatim}
systemctl disable irrigation
\end{verbatim}

When debugging, it may be necessary to temporarily stop the service.  Use this 
command:

\begin{verbatim}
systemctl stop irrigation
\end{verbatim}

To start the service again:

\begin{verbatim}
systemctl start irrigation
\end{verbatim}



