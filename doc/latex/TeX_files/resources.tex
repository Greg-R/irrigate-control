%    Documentation for PRU ADC Project
%    Copyright (C) 2016  Gregory Raven
%
%    This program is free software: you can redistribute it and/or modify
%    it under the terms of the GNU General Public License as published by
%    the Free Software Foundation, either version 3 of the License, or
%    (at your option) any later version.
%
%    This program is distributed in the hope that it will be useful,
%    but WITHOUT ANY WARRANTY; without even the implied warranty of
%    MERCHANTABILITY or FITNESS FOR A PARTICULAR PURPOSE.  See the
%    GNU General Public License for more details.
%
%    You should have received a copy of the GNU General Public License
%    along with this program.  If not, see <http://www.gnu.org/licenses/>.

\chapter{Resources}

\section{Hardware}

\subsection{Beagle Bone Green Wireless}

\url{https://www.seeedstudio.com/SeeedStudio-BeagleBone-Green-Wireless-p-2650.html}

\subsection{Adafruit Proto-Cape}

This allows for a solid interconnect of the BBGW to the solid-state-relay.
The wires are soldered to this board.  The pin numbering is helpful.

\url{https://www.adafruit.com/product/572}

\subsection{SainSmart Solid-State Relay Board}

\url{https://www.amazon.com/gp/product/B00ZZVQR5Q/ref=oh_aui_detailpage_o04_s00?ie=UTF8&psc=1}

There are two interesting points about the various solid-state relay boards.  
Some of them are ``active-low''.  This is bad for this project!  The logic will 
be reversed and the irrigation controls will be ON when the BBGW is OFF!  The 
board in the link above is labelled ``High Level'' and it does indeed use 
conventional high/on low/off logic.

The second point is that the technology in this board is for switching AC 
only!  Beware of specifications being claimed for these devices as the 
information is often incorrect.

\subsection{Power Supplies}

A 5 volt USB supply for the BBGW:

\url{https://www.amazon.com/gp/product/B0117O020U/ref=oh_aui_detailpage_o04_s00?ie=UTF8&psc=1}

This device claims to be extra durable.  It remains to be seen if it will 
survive in a box bolted to the outside of the house.

24VAC supply for the irrigation controls:

\url{https://www.amazon.com/Orbit-Sprinkler-System-Transformer-57040/dp/B000VRYVYS/ref=sr_1_1?ie=UTF8&qid=1497211848&sr=8-1&keywords=orbit+24vac+transformer}

\subsection{Connectors}

These mated JST connectors are perfect for wiring the electronics into the box:

\url{https://www.amazon.com/gp/product/B013WTV270/ref=oh_aui_detailpage_o00_s00?ie=UTF8&psc=1}

\subsection{Irrigation Components}

In general these components are generic and similar products are available from 
different manufacturers.  I chose Orbit and their stuff is holding up so far.

The most important component is the box the BBGW and solid-state relay are 
mounted in.  Since it is mounted on the side of the house is must be sturdy and 
water tight.  This one even includes a ground-interrupt outlet.  The power 
supplies were plugged into this outlet.

\url{https://www.amazon.com/gp/product/B000VYGMF2/ref=oh_aui_detailpage_o05_s00?ie=UTF8&psc=1}

The irrigation pump is a large unit powered by 220VAC.  Originally I was going 
to use a solid-state relay for this.  However, this nicely boxed conventional 
magnetic relay fit into the system nicely and is working great so far:

\url{https://www.amazon.com/gp/product/B000I19I5E/ref=oh_aui_detailpage_o05_s00?ie=UTF8&psc=1}

The ``manifold'' has two 24VAC solenoid valves and made the plumbing easier to 
accomplish.

\url{https://www.amazon.com/Orbit-57250-2-Valve-Preassembled-Manifold/dp/B001H1GWLC/ref=sr_1_1?s=lawn-garden&ie=UTF8&qid=1497212560&sr=1-1&keywords=orbit+2-valve+heavy+duty+preassembled+manifold}

\subsection{Miscellaneous}

The wiring, conduit etcetera is generic and there is no need to list it here.

This assortment of small hardware was very useful:

\url{https://www.amazon.com/gp/product/B01G0LJ8FA/ref=oh_aui_detailpage_o02_s00?ie=UTF8&psc=1}

\section{Software}

\subsection{Github repository for this project}

\url{https://github.com/Greg-R/irrigate-control}

\subsection{WebSockets}

A good introduction to the browser WebSocket API.

\url{https://developer.mozilla.org/en-US/docs/Web/API/WebSockets_API}

\subsection{Beaglebone Green Wireless Development Image}

Recommended stable images are downloaded from this page:

\url{http://beagleboard.org/latest-images}

This project used the IOT non-GUI image:

\url{https://debian.beagleboard.org/images/bone-debian-8.7-iot-armhf-2017-03-19-4gb.img.xz}

\subsection{Node WebSocket Library: ws}

It's easy to use and works great!

\url{https://github.com/websockets/ws}

\subsection{Node Timing Library: node-cron}

This provides the timing functionality used by the Scheduler:

\url{https://github.com/kelektiv/node-cron}

\subsection{Node Mime Parsing Library: mime}

Provides a convenient functionality for detecting mime types in http requests:

\url{https://github.com/broofa/node-mime}

This is used in the server.js code.

\subsection{Node Time Manipulation Library: moment}

\url{https://momentjs.com/}

This improves upon native Javascript date manipulations.  This is used in 
concert with node-cron to implement the scheduling feature.