%    Documentation for PRU ADC Project
%    Copyright (C) 2016  Gregory Raven
%
%    This program is free software: you can redistribute it and/or modify
%    it under the terms of the GNU General Public License as published by
%    the Free Software Foundation, either version 3 of the License, or
%    (at your option) any later version.
%
%    This program is distributed in the hope that it will be useful,
%    but WITHOUT ANY WARRANTY; without even the implied warranty of
%    MERCHANTABILITY or FITNESS FOR A PARTICULAR PURPOSE.  See the
%    GNU General Public License for more details.
%
%    You should have received a copy of the GNU General Public License
%    along with this program.  If not, see <http://www.gnu.org/licenses/>.

\chapter{Configuration the Beagle Bone Green Wireless}

\begin{figure}[h]
	\centering
%    \includegraphics[width=0.7\textwidth]{diagrams/soc_system}
	\centering\bfseries
	\caption{Data Pathways on the AM335X SOC}
\end{figure}

TI has provided example code and kernel drivers for the ``RemoteProc and RemoteProc Messaging Framework''.  A detailed explanation of this framework is available here:

\url{http://processors.wiki.ti.com/index.php/PRU-ICSS_Remoteproc_and_RPMsg}

This framework provides a means of controlling and communicating with the PRUs from user-space, and this project is totally dependent on these functions.

The Remoteproc framework automatically does the job of loading the PRU firmwares from user-space into the PRUs.  Via a sysfs entry, the PRUs can be started and halted from the command line.  These functions are described in the chapter "Shell Scripts".

The examples provided in the PRU Software Support Package show how to use provided functions to send and receive data from PRU to ARM or ARM to PRU.  This is done via character devices which appear in the usual /dev directory.  The standard POSIX functions read/write/open/close work with these character devices.  This allows for typical systems programming technique to be applied when working with the PRUs.

This project uses two character devices assigned to PRU1.  These character devices are used to assign and read back PID parameters to the control system via the user-space executable prumsg.

This project did not require modifications to the loadable kernel modules in the RemoteProc framework.  The modules provided with the IOT Debian-based distribution were used as-is.

\section{The Remoteproc and RPMsg Kernel Modules}

``Loadable Kernel Modules'' (LKMs) must be active for this project to function.
At the shell command line, execute this command:

\begin{verbatim}
lsmod
\end{verbatim}

This will list the LKMs currently loaded.  The modules associated with Remoteproc are:

\begin{itemize}
\item pruss
\item pru\_rproc
\item pruss\_intc
\end{itemize}

There are two modules associated with RPMsg, and these will appear in the list only after the firmwares are loaded into the PRUs:

\begin{verbatim}
virtio_rpmsg_bus
rpmsg_pru
\end{verbatim}

The Remoteproc kernel modules may not be loaded at boot (depending on boot configuration).  However, they can be loaded (or unloaded) anytime after boot with the following commands:

\begin{verbatim}
modprobe pru_rproc
rmmod pru_rproc
\end{verbatim}

``modprobe'' is the load command; ``rmmod'' is the remove command.

Perhaps a better method of controlling the Remoteproc framework is via the sys virtual file system.  A set of shell scripts is included in the repository which includes these commands in the ``prumodout'' and ``prumodin'' scripts.

\section{Files Associated with RemoteProc in the Compilation Process}

There are some interesting files in the root directory of the github repository for this project:

\begin{itemize}
\item resource\_table\_0.h
\item resource\_table\_1.h
\item AM335x\_PRU.cmd
\end{itemize}

These files were copied verbatim from the PRU Support Package.

Jason Reeder of Texas Instruments has provided an explanation of the files required to compile firmwares for the PRUs:

\begin{quotation}
There are four files needed in order to build a C project for the PRU using TI's C compiler. Each of the examples and labs in the pru-software-support-package include these files:
\begin{enumerate}

\item  yourProgramFile.c

This is your C program that you are writing for the PRU.

\item   AM335x\_PRU.cmd

This is a command linker file. This is the way that we describe the physical memory map, the constant table entries, and the placement of our code and data sections (into the physical memory described at the top of the file) to the PRU linker. There are some neat things that can be done as far as placing code and data in exact memory locations using this file, but for the majority of projects, this file can remain unchanged.

\item   resource\_table\_*.h

This file will create a header in the elf binary (generated .out file) that is needed by the RemoteProc Linux driver while loading the code into the PRUs. For the examples in the pru-software-support-package, there are two types of resources that the PRU can request using the resource\_table\_*.h file:

interrupts - letting RemoteProc configure the PRU INTC interrupts saves code space on the PRUs.

vrings - requesting vrings in the resource\_table file is necessary if rpmsg communication is desired (since the ARM/Linux needs to create the vrings in DDR and then notify the PRU where the vrings were placed)
even if no resources are needed, the RemoteProc Linux driver expects the header to exist. Because of this, many examples in the package contain an empty resource\_table header file (resource\_table\_empty.h)

\item   Makefile

Makefile to build your PRU C program either on the target, on your Linux machine, or even on a Windows machine. The comment at the top of the Makefile tries to explain the environment variable needed for a successful build and how to set it on each of the three supported build development environments.  (Note that this is in reference to the Makefiles provided with the PRU Support Package.  The custom Makefile provided with this project does not have this feature.)

\end{enumerate}
\end{quotation}







