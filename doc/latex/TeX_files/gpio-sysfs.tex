%    Documentation for Irrigation Control Project
%    Copyright (C) 2017  Gregory Raven
%
%    This program is free software: you can redistribute it and/or modify
%    it under the terms of the GNU General Public License as published by
%    the Free Software Foundation, either version 3 of the License, or
%    (at your option) any later version.
%
%    This program is distributed in the hope that it will be useful,
%    but WITHOUT ANY WARRANTY; without even the implied warranty of
%    MERCHANTABILITY or FITNESS FOR A PARTICULAR PURPOSE.  See the
%    GNU General Public License for more details.
%
%    You should have received a copy of the GNU General Public License
%    along with this program.  If not, see <http://www.gnu.org/licenses/>.

\chapter{GPIO Control with sysfs Virtual File System}

This ``sysfs'' virtual file system the core functionality which allows the 
GPIOs to be controlled from user space.

Note that the BBGW must be correctly configured for GPIO output mode on the 
three control pins used to control the irrigation devices.  This document will 
not cover this subject in detail, as this has been well-covered in numerous web 
articles and books.

A highly recommended resource is ``Exploring Beaglebone'' by Derek Molloy.
(add footnote here)

The method of configuring the header pins to GPIO is covered in the chapter 
``Universal IO''.  GPIO configuration must be complete before any of the 
commands shown below will function properly.

``POSIX'' type operating systems, which includes Linux, are ``file based''.  
That means the interface to everything is via writing to or reading from a 
file.  In this case, the GPIO's state is changed by writing a 0 or 1 to the 
appropriate file.  Here is an example:

\begin{verbatim}
echo 1 > /sys/class/gpio/gpio50/value
\end{verbatim}

The above changes the state of header pin P9.14 to ``high'' or an output of 3.3 
volts.  Echoing 0 changes the output to 0.0 volts.  It's that simple!

\subsection{Controlling the GPIOs using Javascript}

The command shown above is typed and executed in a bash shell.  How is this 
done from Javascript?  A module from Node.js is used to accomplish this:

\url{https://nodejs.org/api/child_process.html}

For example, the bash command shown above would be executed as follows in 
Javascript:

\begin{verbatim}
const exec = require('child_process').exec;
exec('echo 1 > /sys/class/gpio/gpio50/value');
\end{verbatim}

This is the simplest possible usage; an optional callback function as a second 
option is possible.  The callback option is used in the in the ``pumpActuator'' 
class, and the callback is used to emit an event from the pumpActuator Object.  
This event is subscribed to by the WebSocket server, and when the event fires 
it sends a message to the web page controller to indicate that the control 
function has changed state.  The web page is updated to indicate the new state.

The class method looks like this:

\begin{verbatim}
    pumpControl(pumpgpio, command) {
        const exec = require('child_process').exec;
        exec(`echo ${command} > ${this.pumpGpioMap.get(pumpgpio)}`, (error, 
        stdout, stderr) => {
            // If error, do not update the status of the controls.
            if (error) {
                console.error(`exec error: ${error}`);
                return;
            } else {
                console.log(`Status message emitted from ledActuator: 
                ${pumpgpio} is set to ${command}.`);
                //  Send a JSON object with the value being an array.
                this.emit('statusmessage', `["${pumpgpio}",${command}]`);
            }
        });
    }
\end{verbatim}







