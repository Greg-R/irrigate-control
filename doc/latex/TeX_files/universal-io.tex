%    Documentation for PRU ADC Project
%    Copyright (C) 2016  Gregory Raven
%
%    This program is free software: you can redistribute it and/or modify
%    it under the terms of the GNU General Public License as published by
%    the Free Software Foundation, either version 3 of the License, or
%    (at your option) any later version.
%
%    This program is distributed in the hope that it will be useful,
%    but WITHOUT ANY WARRANTY; without even the implied warranty of
%    MERCHANTABILITY or FITNESS FOR A PARTICULAR PURPOSE.  See the
%    GNU General Public License for more details.
%
%    You should have received a copy of the GNU General Public License
%    along with this program.  If not, see <http://www.gnu.org/licenses/>.

\chapter{Universal IO}

This project requires only three GPIOs.  Rather than editing device tree files, 
and having to deal with potential bugs caused by this, the excellent config-pin 
utility was used.  This utility is provided by the Universal IO project.

The Universal IO project is located at this Github repository:

\url{https://github.com/cdsteinkuehler/beaglebone-universal-io}

Universal IO is included with the most recent Debian-based IOT images.

The setting of the three GPIOs is done by the pumpActuator class.  This is most 
appropriate, as the class uses these GPIOs to control the irrigation devices.  
The GPIO configuration is done by the class constructor:

\begin{verbatim}
    setGpio(headerPin) {
        const exec = require('child_process').exec;
        console.log(`Setting header pin ${headerPin} to GPIO mode.`);
        exec(`config-pin ${headerPin} low_pd`);
    }
\end{verbatim}

Note the use of the Node.js ``child\_process'' as described in the chapter on 
GPIO control with sysfs.

The three pins are set to GPIO low state and pull-down mode via these three 
lines in the class constructor using the above method:

\begin{verbatim}
        this.setGpio('P9.14');
        this.setGpio('P9.15');
        this.setGpio('P9.16');
\end{verbatim}

This partitioning of the GPIO mode set functioning into the pumpActuator class 
is appropriate and eliminated the requirement for an auxiliary shell script for 
control of header pin modes.




