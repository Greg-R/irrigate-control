%    Documentation for PRU ADC Project
%    Copyright (C) 2016  Gregory Raven
%
%    This program is free software: you can redistribute it and/or modify
%    it under the terms of the GNU General Public License as published by
%    the Free Software Foundation, either version 3 of the License, or
%    (at your option) any later version.
%
%    This program is distributed in the hope that it will be useful,
%    but WITHOUT ANY WARRANTY; without even the implied warranty of
%    MERCHANTABILITY or FITNESS FOR A PARTICULAR PURPOSE.  See the
%    GNU General Public License for more details.
%
%    You should have received a copy of the GNU General Public License
%    along with this program.  If not, see <http://www.gnu.org/licenses/>.

\chapter{Device Tree Requirements}

This project requires a custom ``Device Tree Include''.  This is a device tree fragment which is inserted into the top device tree file.  The dtsi directory located in the software directory contains the file and and a README file which explains how to edit the device tree source file.

The same file, which in this case is ``am335x-bonegreen.dts'', must be modified in order to active the RemoteProc framework kernel drivers.  It is recommended to add the include statement at the same time the RemoteProc is activated.  This step is included in the RemoteProc and PRU Compiler step-by-step process in Chapter 9.

There is one PRU GPIO output enabled on header P9 and this is used to monitor Quadrature Decoder under/overflow.  This configuration is done in the same file ``pru\_gpio\_config'' which is sourced by .bashrc as discussed in Chapter 6.

The Universal IO project is located at this Github repository:

\url{https://github.com/cdsteinkuehler/beaglebone-universal-io}

Universal IO is included with the most recent Debian-based IOT images.




