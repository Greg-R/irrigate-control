%    Documentation for Irrigation Control Project
%    Copyright (C) 2017  Gregory Raven
%
%    This program is free software: you can redistribute it and/or modify
%    it under the terms of the GNU General Public License as published by
%    the Free Software Foundation, either version 3 of the License, or
%    (at your option) any later version.
%
%    This program is distributed in the hope that it will be useful,
%    but WITHOUT ANY WARRANTY; without even the implied warranty of
%    MERCHANTABILITY or FITNESS FOR A PARTICULAR PURPOSE.  See the
%    GNU General Public License for more details.
%
%    You should have received a copy of the GNU General Public License
%    along with this program.  If not, see <http://www.gnu.org/licenses/>.

\chapter{Introduction}

This is the documentation for an embedded GNU/Linux project using the 
Beaglebone Green Wireless (BBGW) development board.  The project repository is 
located here:

\url{https://github.com/Greg-R/irrigate-control}

A listing of parts and other resources is found in the Resources chapter.

\section{Project Goals}

I've published two other projects on Hackster.io.  These other projects are 
interesting, and good learning tools, but they do not do anything useful in the 
real world.  After a couple of years of "going up the learning curve" on 
embedded device development, it was time to create something practical!

This project performs the simple home automation task of controlling a lawn 
irrigation system.  For many years, I had been controlling the irrigation 
system manually by flipping a circuit breaker to turn the pump motor on and 
off.  The old timing unit had failed years ago, and I had never replaced it.  
I wanted to be able to control the system via a web browser without 
having to leave my home office desk.

Although in principle the automation is simple, the underlying technology is 
complex!  There was a significant investment in time learning the technology 
required to implement the project.

\section{Technologies}

The development board chosen is the Beagle Bone Green Wireless (BBGW). The 
board will be mounted remotely 
with access to power only (no wired ethernet is possible at that location).

The Debian-based GNU/Linux distribution used on the BBGW can be downloaded from 
this page:

\url{http://beagleboard.org/latest-images}

The ``IOT'' (non-GUI) image was chosen, as this provides most of the software 
features required to get the project up and running.  As is, the board is 
configured as an ``access point''.  This was changed to a conventional wireless 
LAN interface.

Node.js version v8.1.0 was used.  This is a much later version than what is 
included in the image.  A section on upgrading Node is included.

The project's Javascript code uses several ``Ecmascript 6'' constructs.  In my 
opinion this release of Javascript is a significant improvement in the 
language.  Some of the strange quirks of Javascript are eliminated!

Two-way communication between the web browser and the BBGW was done with 
``WebSockets``.  The client side WebSocket is built into the browser.  Using 
the latest updates in Ubuntu 16.04, both Firefox and Chromium browsers include 
this capability.  The Chrome browser of an Android phone was also found to work.

Here is a good reference on the client (browser) side WebSockets:

\url{https://developer.mozilla.org/en-US/docs/Web/API/WebSocket}

The server side uses a Node.js package called ``ws'':

\url{https://github.com/websockets/ws}

Since this is a ``real world'' project, there has to be an interface with real 
actuators.  This small ``solid-state-relay'' board proved to be ideal for 
taking the GPIO outputs of the BBGW and doing something real:

\url{https://www.amazon.com/gp/product/B00ZZVQR5Q/ref=oh_aui_detailpage_o00_s00?ie=UTF8&psc=1}

This board has four individual relays.  This project uses only three.

Note that this type of relay can switch AC power only.  This was ideal for this 
project, as the irrigation system controls are powered by 24VAC.

The other common components of the irrigation system are listed in the 
reference section at the end of this document.

``Universal IO'' was used to set the pin multiplexer to GPIO mode:

\url{https://github.com/cdsteinkuehler/beaglebone-universal-io}

The irrigation system control components were manufactured by Orbit.  These 
types of solenoid valves and relays seem to be standardized to 24VAC power, so 
other manufacturers products will work as well.




