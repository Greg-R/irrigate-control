%    Documentation for PRU ADC Project
%    Copyright (C) 2016  Gregory Raven
%
%    This program is free software: you can redistribute it and/or modify
%    it under the terms of the GNU General Public License as published by
%    the Free Software Foundation, either version 3 of the License, or
%    (at your option) any later version.
%
%    This program is distributed in the hope that it will be useful,
%    but WITHOUT ANY WARRANTY; without even the implied warranty of
%    MERCHANTABILITY or FITNESS FOR A PARTICULAR PURPOSE.  See the
%    GNU General Public License for more details.
%
%    You should have received a copy of the GNU General Public License
%    along with this program.  If not, see <http://www.gnu.org/licenses/>.

\chapter{Introduction}

This is the documentation for an embedded GNU/Linux project utilizing the RemoteProc and RPMsg framework in the Beaglebone Green (BBG) development board.  The project repository is located here:

\url{https://github.com/Greg-R/pru-pid-motor}

The inspiration for this project came from an example project published by Texas Instruments.  The Texas Instruments project is based on "Code Composer Studio", which is an "Integrated Development Environment" (IDE):

\url{http://processors.wiki.ti.com/index.php/PRU_Training:_PRU_PID_Motor_Demo}

There is also a PDF file which describes the project in detail:

\url{http://www.ti.com/lit/ug/tidubj6/tidubj6.pdf}

The TI project requires a relatively complex cross-compiler installation.  This project is designed to be done via SSH terminal connection, and all software compilation was done on the Beaglebone Green target device using the PRU C compiler (clpru).  The VIM text editor was used to edit files, however, any text editor available in the Debian distribution can be used.

The Debian-based GNU/Linux distribution used on the BBG can be downloaded from this page:

\url{http://beagleboard.org/latest-images}

The ``IOT'' (non-GUI) image was chosen, as this provides the shortest path to get the project up and running.

Recent developments in the Texas Instruments PRU support include the RemoteProc and Remote Messaging frameworks, as well as an extensively documented C compiler and much additional supporting documentation.  This project utilizes these frameworks and is entirely dependent upon C code in both the PRU and GNU/Linux user space.  For further information, refer to the detailed examples provided by TI in the ``PRU Support Package'':

\url{https://git.ti.com/pru-software-support-package}

A listing of additional resources is found in the Resources chapter.

The motor recommended by TI was purchased and tested.  However, a better motor with an integrated quadrature encoder was found on eBay and is recommended.  A chapter is included which describes this motor-encoder and how to obtain one.

\section{Project Goals}

This project demonstrates an electronic speed control for a DC motor which is implemented with the PRUs included with the Beaglebone Green.  Beyond its usefulness as a demonstration project, it could be used in a robotics project such as a ``mobile robot''.

The basic principle of the speed controller is "Proportional Integral Derivative" (PID) feedback control, which is a common feedback controller used in digital systems.

Here is an excellent reference article on PID controllers:

\url{http://www.wescottdesign.com/articles/pid/pidWithoutAPhd.pdf}

\section{Limitations}

All of the development was done as root user via ssh on the BeagleBone Green.  This is generally not a good practice, however, considering this as an embedded and experimental project it was not considered to be a serious drawback.

No attempt was made to optimize the response of the PID controller.  The default values for the PID controller create a stable loop with the recommended DC motor-encoder.  Optimization will depend on the particular motor-encoder chosen and this task is left to the interested experimenter.


